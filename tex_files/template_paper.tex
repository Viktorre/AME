\documentclass[11pt]{article}
\usepackage[hmargin=2cm,vmargin=2cm]{geometry}
\usepackage{natbib}
\usepackage[english]{babel}
\usepackage{graphicx}
\usepackage{amstext}
\usepackage{setspace}
\usepackage{threeparttable}
\usepackage{subfigure}
\usepackage{booktabs}
\usepackage{rotating}
\usepackage{float}
\usepackage{fancybox}
\usepackage{xcolor}
\usepackage{array,color}
\usepackage{dcolumn}
\usepackage[font=small,labelfont=bf]{caption}
\usepackage{longtable}
\usepackage{verbatim}
\usepackage{pdflscape}
\usepackage{amssymb}
\usepackage{lscape}
\usepackage{tabularx}
\usepackage{verbatim}
\usepackage{geometry}
\usepackage{bbold}
\usepackage{apalike}
%\usepackage{hyperref}
\usepackage{mathtools}
\usepackage{amsmath}
\usepackage{color,soul}
\usepackage{bibentry}
\usepackage{pxfonts}
\usepackage{palatino}
\usepackage{authblk}
\usepackage{enumitem}
\usepackage[toc,page]{appendix}
\usepackage{chngcntr}
\usepackage[utf8]{inputenc}
\usepackage{pdflscape}
\setlength{\parindent}{1cm}
\setlength{\parskip}{2mm}




%%%%%%%%%%%%%%%%%%%%%%%%%%%%%%%%%%%%%%%%%%%%%%%%%%%%%%%%%%%%%%
\title{Recreation of Temperature and Decisions: Evidence from 207,000
	Court Cases}
\author{Viktor Reif}
\date{\today}


\begin{document}
\maketitle


%%%%%%%%%%%%%%%%%%%%%%%%%%%%%%%%%%%%%%%%%%%%%%%%%%%%%%%%%%%%%%
% Abstract
%%%%%%%%%%%%%%%%%%%%%%%%%%%%%%%%%%%%%%%%%%%%%%%%%%%%%%%%%%%%%%
\begin{abstract}
	\singlespacing
	\noindent 
	0. Abstract: summarize the key points of the paper.
	\newline \noindent \textbf{Keywords:} xxx; xyz; zyx. 
\end{abstract}


%%%%%%%%%%%%%%%%%%%%%%%%%%%%%%%%%%%%%%%%%%%%%%%%%%%%%%%%%%%%%%
% Section 1: 
%%%%%%%%%%%%%%%%%%%%%%%%%%%%%%%%%%%%%%%%%%%%%%%%%%%%%%%%%%%%%%
\section{Introduction}
introduce the topic and state the aim of your work, stating clearly the
research questions and the methodology used, give a brief overview of the results and the
limitations of your analysis.
a what my paper does, b what underlying aper does (and why relevant?), c what i can confirm reject
\newline This paper examines the robustness of the results in the article “Temperature and Decisions: Evidence from 207,000” by Heyes and Saberian in 2018. Using the same dataset, this paper recreates the main findings. The aim of this paper is to either empirically confirm the results of Heyes and Saberian or to disprove them and illuminate the reasons for that. I reimplement the entire analysis in Python and add a few more model specifications to further increase the validity of the results.
\newline Heyes and Saberian use a dataset of 207,000 court cases in the U.S. for a holistic regression analysis to evluate the influence of climate on professionally made decisions. QUOTE HERE The authors use a holistic set of explanatory variables including various fixed effects - over time, across judges and locations, etc - to control for heterogeneity in the regression of court case outcome on temperature. In this analysis they find a significant relation between temperature and case outcome.
\newline I am able to replicate the paper’s main finding. In my analysis, estimated coefficients differ in values but not in direction and significance. The added specifications omitted by the original paper further confirm the underlying relation between climate and decisions. Moreover, I contribute a more accessible technical infrastructure in Python for other researchers to also replicate these results.  
%%%%%%%%%%%%%%%%%%%%%%%%%%%%%%%%%%%%%%%%%%%%%%%%%%%%%%%%%%%%%%
% Section 2: 
%%%%%%%%%%%%%%%%%%%%%%%%%%%%%%%%%%%%%%%%%%%%%%%%%%%%%%%%%%%%%%
\section{Literature review}
summarize the current state of knowledge on the topic of your paper
(including the latest relevant publications).
\subsection{Title of subsection in Section 2}
afasdasfasfasd (\cite{Landrigan.2017}). asdla (\cite{Test.01012022})
\subsection{Title of subsection in Section 2}
afasdasfasfasd.
%%%%%%%%%%%%%%%%%%%%%%%%%%%%%%%%%%%%%%%%%%%%%%%%%%%%%%%%%%%%%%
% Section 3: 
%%%%%%%%%%%%%%%%%%%%%%%%%%%%%%%%%%%%%%%%%%%%%%%%%%%%%%%%%%%%%%
\section{Data}
: Describe the main sources of your data, the data cleaning and merging process,
include a table(s) of summary statistics and a brief description of these.
%%%%%%%%%%%%%%%%%%%%%%%%%%%%%%%%%%%%%%%%%%%%%%%%%%%%%%%%%%%%%%
% Section 4: 
%%%%%%%%%%%%%%%%%%%%%%%%%%%%%%%%%%%%%%%%%%%%%%%%%%%%%%%%%%%%%%
\section{Empirical strategy}
describe the empirical method used and its appropriateness in this
context, state the main hypotheses to be tested.
%%%%%%%%%%%%%%%%%%%%%%%%%%%%%%%%%%%%%%%%%%%%%%%%%%%%%%%%%%%%%%
% Section 5: 
%%%%%%%%%%%%%%%%%%%%%%%%%%%%%%%%%%%%%%%%%%%%%%%%%%%%%%%%%%%%%%
\section{Results}
present and comment on your results.

\begin{tabular}{lllll}
	\toprule
	{} &        (1) &       (2) &       (3) &        (4) \\
	\midrule
	&  Preferred &     1-Day &       lag &      1-Day \\
	Temperaturet/1000                 &  -1.363*** &  1.108*** &  -2.21*** &     -0.278 \\
	&    [0.155] &   [0.211] &   [0.262] &    [0.352] \\
	Temperaturet-1/1000               &          - &  -2.17*** &         - &  -1.866*** \\
	&          - &   [0.174] &         - &    [0.228] \\
	Temperaturet+1/1000               &          - &         - &  0.818*** &   1.022*** \\
	&          - &         - &         - &          - \\
	F-statistic of joint significance &   2417.651 &  2268.988 &  2257.756 &   2121.681 \\
	of weather variables              &            &           &           &            \\
	P-value                           &      0.000 &     0.000 &     0.000 &      0.000 \\
	\midrule
	Observations                      &     169006 &    169006 &    169006 &     169006 \\
	\bottomrule
\end{tabular}


%%%%%%%%%%%%%%%%%%%%%%%%%%%%%%%%%%%%%%%%%%%%%%%%%%%%%%%%%%%%%%
% Section 6: 
%%%%%%%%%%%%%%%%%%%%%%%%%%%%%%%%%%%%%%%%%%%%%%%%%%%%%%%%%%%%%%
\section{Discussion}
reflect on the meaning and policy implications of your results, think of potential
limitations to your work and avenues for future research.
%%%%%%%%%%%%%%%%%%%%%%%%%%%%%%%%%%%%%%%%%%%%%%%%%%%%%%%%%%%%%%
% Section 7: 
%%%%%%%%%%%%%%%%%%%%%%%%%%%%%%%%%%%%%%%%%%%%%%%%%%%%%%%%%%%%%%
\section{ Conclusion}
summarize your main work and conclude.
1. Select a paper that uses one of the empirical methods reviewed in class
2. Get the raw data
3. Replicate the data analysis
4. Write a report summarizing your work.
5. Include a literature review section in your report that summarizes the current state
of knowledge on your topic.



% % % % % % % % % % % % % % % % % % % % % % % % % % % % % % % % % %
% Appendix
% % % % % % % % % % % % % % % % % % % % % % % % % % % % % % % % % %
\newpage

\begin{subappendices}
\counterwithin{figure}{section}
\counterwithin{table}{section}
\counterwithin{equation}{section}
\appendix

%%%%%%%%%%%%%%%%%%%%%%%%%%%%%%%%%%%%
\section*{Appendix}\label{Appendix}
%%%%%%%%%%%%%%%%%%%%%%%%%%%%%%%%%%%%%%
DLETE THIS LATER use it for additional material that might support your analysis + (in the nal
version) include a separate paragraph that provides a response to the referee's comments
and mentions where, how, why, why not the paper has changed.
\singlespacing
\section{Summary statistics}\label{ASec:xxxxx}
	
\end{subappendices}	



%%%%%%%%%%%%%%%%%%%%%%%%%%%%%%%%%%%%%%%%%%%%%%%%%%%%%%%%%%%%
% References
%%%%%%%%%%%%%%%%%%%%%%%%%%%%%%%%%%%%%%%%%%%%%%%%%%%%%%%%%%%%
\newpage
{\footnotesize 
	\bibliographystyle{apalike}
	\singlespacing
	\bibliography{references2.bib}
}


%%%%%%%%%%%%%%%%%%%%%%%%%%%%%%%%%%%%%%%%%%%%%%%%%%%%%%%%%%%%
\end{document}