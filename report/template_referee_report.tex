\documentclass[11pt]{article}
\usepackage[hmargin=2cm,vmargin=2cm]{geometry}
\usepackage{natbib}
\usepackage[english]{babel}
\usepackage{graphicx}
\usepackage{amstext}
\usepackage{setspace}
\usepackage{threeparttable}
\usepackage{subfigure}
\usepackage{booktabs}
\usepackage{rotating}
\usepackage{float}
\usepackage{fancybox}
\usepackage{xcolor}
\usepackage{array,color}
\usepackage{dcolumn}
\usepackage[font=small,labelfont=bf]{caption}
\usepackage{longtable}
\usepackage{verbatim}
\usepackage{pdflscape}
\usepackage{amssymb}
\usepackage{lscape}
\usepackage{tabularx}
\usepackage{verbatim}
\usepackage{geometry}
\usepackage{bbold}
\usepackage{apalike}
%\usepackage{hyperref}
\usepackage{mathtools}
\usepackage{amsmath}
\usepackage{color,soul}
\usepackage{bibentry}
\usepackage{pxfonts}
\usepackage{palatino}
\usepackage{authblk}
\usepackage{enumitem}
\usepackage[toc,page]{appendix}
\usepackage{chngcntr}
\usepackage[utf8]{inputenc}
\usepackage{pdflscape}
\setlength{\parindent}{1cm}
\setlength{\parskip}{2mm}



%%%%%%%%%%%%%%%%%%%%%%%%%%%%%%%%%%%%%%%%%%%%%%%%%%%%%%%%%%%%%%
\title{Referee report for: “Environmental Regulations, Air and Water
	Pollution, and Infant Mortality in India”, Tom Schmid}
\author{By: Viktor Reif}
\date{\today}


\begin{document}
	\maketitle
	
	


%%%%%%%%%%%%%%%%%%%%%%%%%%%%%%%%%%%%%%%%%%%%%%%%%%%%%%%%%%%%%%
% Section 1: 
%%%%%%%%%%%%%%%%%%%%%%%%%%%%%%%%%%%%%%%%%%%%%%%%%%%%%%%%%%%%%%
\section{Summary}
In an effort to replicate the empiric analysis in Greenstone and Hanna (2014), Tom uses a difference-in-difference regression approach to find the effects of three policies on the environment and health in city-time panel dataset about India. This analysis yields no significant effects for an anti water-pollution policy on the environment. The other two policies are air quality policies and have a significant effect on the environment but not on health. His results are generally in line with the findings by Greenstone and Hanna.
\newline Apart from successfully replicating the empiric results from the original paper, Tom manages to give great insight into the theoretical context (literature in the introduction) as well as the context of the data used in the analysis.  
Moreover, thanks to the precise and short abstract, results and summary chapters, it is easy to identify what the main findings of these paper are. 
The method chapter describes the used method thoroughly and in an understandable way, without confusing the reader. 
Also, the interpretations in the results chapter are written in a clear way which help the general understanding of the results. 
\newline A main problem of the paper is that is not evident from the beginning who (Greenstone/Hanna and Tom) is doing what. In the abstract or at the latest in the introduction it should be clear that this paper tries to recreate the empirical analysis of said underlying paper and what the results of both papers are. This is not the case. 
Tom merges his literature review into the introduction. A shortcoming here is that even though Tom gives an exhaustive (yet scattered) overview of literature about climate change, pollution, the situation in India, etc., it does not summarize the current state of knowledge for his key topic. The key topic is impact of environmental policies on pollution levels and human health. For that Tom does not showcase any literature for India or other countries. 
Also, the introduction does not precisely specify the research question from the abstract (i.e. impact of policies on environment and health). In the introduction Tom suggests assessing the relation between policies and environment and the relation between environment and health. It is not clear which is the dependent variable. In the abstract it is health and environment, in the introduction only health. Is environment a kind of intermediary regressor between policies and health? Generally, the introduction fails to explain what the intuition of the underlying model (without any details) is meant to be. 
In the methods the choice of empirical method is not motivated in any way and it is not argued why the chosen method is useful in this context.
The discussion chapter does not consider possible implications of the found results and does not suggest avenues for further research.


%%%%%%%%%%%%%%%%%%%%%%%%%%%%%%%%%%%%%%%%%%%%%%%%%%%%%%%%%%%%%%
% Section 2: 
%%%%%%%%%%%%%%%%%%%%%%%%%%%%%%%%%%%%%%%%%%%%%%%%%%%%%%%%%%%%%%
\section{General comments}
The analysis seems coherent both in structure and in its implementation. Its results are in line with the theory presented at beginning and the underlying paper. Deviations from Greenstone and Hanna are consistently commented as such along with a plausible explanation. In most cases deviations arose due to model specification issues, which is acceptable for a seminar paper. I must mention however, that “possible differences in the numerical computation of the regression” as mentioned once by Tom, is not an adequate justification for deviations.
\newline The argumentation is also consistently plausible. Apart from the mentioned confusion in the introduction chapter, Tom proposes his research questions, works them in the analysis and thus empirically confirms/denies the relations/effects he initially suggested through theory. A rather small issue in the argumentation that this analysis serves as an example for effective environment policies under weak institutions. This requires several assumptions beforehand, such as India being institutionally weak and a relation between institution quality and policy effectiveness. These assumptions are implied, but not proven empirically or through literature.

%%%%%%%%%%%%%%%%%%%%%%%%%%%%%%%%%%%%%%%%%%%%%%%%%%%%%%%%%%%%%%
% Section 3: 
%%%%%%%%%%%%%%%%%%%%%%%%%%%%%%%%%%%%%%%%%%%%%%%%%%%%%%%%%%%%%%
\section{Minor comments}
The following are small comments about details in no specific order. On several occasions Tom gives too many details, which are partly redundant. The use of literature was partly (in chapter 2 for justifying the relevance of variables, in chapter 5 for explaining the differences in effectiveness of policies) too exhaustive. Some descriptions are too detailed (e.g. Figure 1 and table 2). The description of Table 1 is incomplete and therefore confusing (I compared it to the description in the original paper). Figure 1 would be easier to read with percent changes for all y-axes to have comparable scales. Figure 3 has wrong/missing headings. All tables lack asterisks too indicate significance and contain unnecessary decimals in some rows (e.g. Observations). The quantity of tables and figures is large enough to put all but the most important ones into the appendix to increase readability. The abstract does not have keywords. There are some typos.


%%%%%%%%%%%%%%%%%%%%%%%%%%%%%%%%%%%%%%%%%%%%%%%%%%%%%%%%%%%%
\end{document}